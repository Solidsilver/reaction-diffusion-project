\documentclass[t,10pt,fleqn]{beamer}

%%% My preferred theme choices.  More information is available at
%
%  https://en.wikipedia.org/wiki/Beamer_(LaTeX)
%  http://mirrors.ctan.org/macros/latex/contrib/beamer/doc/beameruserguide.pdf

\usetheme{Malmoe}
\usecolortheme{rose}
\useinnertheme{rounded}
\useoutertheme[subsection=false,footline=authorinstitute]{miniframes}
\usenavigationsymbolstemplate{}

%%% The following command inserts a slide with the outline at the
%%% beginning of each section, highlighting the current section.  It
%%% is optional.

\AtBeginSection[]
{
  \begin{frame}{Outline}
    \tableofcontents[currentsection]
  \end{frame}
}

\newenvironment{amatrix}[1]{%
  \left(\begin{array}{@{}*{#1}{r}|r@{}}
}{%
  \end{array}\right)
}

\newenvironment{bulletlist}
   {
      \begin{list}
         {$\bullet$}
%         {$\cdot$}
         {
            \setlength{\itemsep}{.5ex}
            \setlength{\parsep}{0ex}
            \setlength{\leftmargin}{1.5 em}
            \setlength{\rightmargin}{0.5em}
            \setlength{\parskip}{0ex}
            \setlength{\topsep}{0ex}
         }
   }
   {
      \end{list}
   }


\newcount\arrowcount
\newcommand\arrows[1]{
        \global\arrowcount#1
        \ifnum\arrowcount>0
                \begin{matrix}
                \expandafter\nextarrow
        \fi
}

\newcommand\nextarrow[1]{
        \global\advance\arrowcount-1
        \ifx\relax#1\relax\else \xrightarrow{#1}\fi
        \ifnum\arrowcount=0
                \end{matrix}
        \else
                \\
                \expandafter\nextarrow
        \fi
}


%%% It is sometimes easier to have graphics in a subfolder (or
%%% subfolders) of the current folder, in this case that folder is
%%% called Figures
\graphicspath{ 
  {Figures/}
}


\usepackage{multicol}
  \usepackage{booktabs}
  \usepackage{amsmath}
\usepackage{epsfig}
\usepackage{enumerate}

\def\ds{\displaystyle}
\def\u{{\mathbf u}}
\def\v{{\mathbf v}}
\def\w{{\mathbf w}}

\def\A{{\mathbf A}}



\def\B{{\mathbf B}}
\def\C{{\mathbf C}}
\def\D{{\mathbf D}}
\def\E{{\mathbf E}}
\def\X{{\mathbf X}}
\def\U{{\mathbf U}}

\def\T{{\mathbf T}}

\def\det{\text{det}}
\def\row{\text{row}}
\def\col{\text{col}}
\def\dim{\text{dim}}
\def\span{\text{span}}
\def\rank{\text{rank}}
\def\dom{\text{dom}}
\def\domain{\text{domain}}
\def\range{\text{range}}
\def\RREF{\text{RREF}}
\def\null{\text{null}}
\def\nullity{\text{nullity}}
\def\ker{\text{ker}}


\def\e{{\mathbf e}}
\def\x{{\mathbf x}}
\def\y{{\mathbf y}}
\def\b{{\mathbf b}}
\def\c{{\mathbf c}}

\def\r{{\mathbf r}}

\def\0{{\mathbf 0}}
\def\v{{\mathbf v}}
\def\I{{\mathbf I}}


\def\AA{\mathcal{A}}
\def\R{\mathcal{R}}

\def\S{\mathcal{S}}

\def\V{\mathcal{V}}

\def\M{\mathcal{M}}

\def\({\biggr ( }
\def\){\biggr ) }

\def\[{\biggr [ }
\def\]{\biggr ] }

\def\d{\partial}

\newcommand{\tu}[1]{\underline{\textit{#1}}}

% This is the main file
% This is the main file

\title[Reaction-Diffusion Equations]%
      {Reaction-Diffusion Equations}
\subtitle{Turing Mechanisms and Morphogenesis}
\author[Luke Mattfeld]{Luke Mattfeld}
\institute[EWU]{Eastern Washington University}
\date{March 13, 2020}

\begin{document}

\begin{frame}
\titlepage

\end{frame}
%-------------------------------------------------------------------------------------------------------------------------------------
%-------------------------------------------------------------------------------------------------------------------------------------
\section{Background}
%------------------------------------------------------------------------------------------------------------------------------
\begin{frame}{Morphogenesis}
\vspace{-.3cm}
\begin{block}{}
\begin{itemize}
    \pause
      \item Class consists of three core topics:
               \pause
               \begin{enumerate}
                    \pause
                    \item Model Construction
                       \pause
                    \item Solving PDEs on unbounded domains
                     \pause
                       \item Solving PDEs on bounded domains
                \end{enumerate}
          \pause
     \end{itemize}
  \end{block}
   
   \begin{block}{Grade}
\begin{itemize}
    \pause
      \item Class grade is determined from:
               \pause
               \begin{itemize}
                    \pause
                    \item 2 exams
                     \pause
                     \item Quizzes = 0.5 exam
                     \pause
                    \item Written HW = 1.5 exam   \pause 
                    (one-sided, neat, organized, cover page)
                     \pause
                    \item Course Project and Presentation = 1 exam  

                \end{itemize}
     \end{itemize}
  \end{block}
   
 \end{frame}
 
%-------------------------------------------------------------------------------------------------------------------------------------
%------------------------------------------------------------------------------------------------------------------------------
\begin{frame}

   \begin{block}{}
  \begin{itemize}
    \pause
      \item Predominantly, this class involves the construction and  analysis of PDE models
      \pause
       \item     Convenient software packages to check your work include:  MATLAB, Octave (similar to MATLAB but free), Julia, SAGE, or  Python (with its associated libraries)
      \pause
      \item Software won't help if one has the wrong model.
      \pause
     \end{itemize}
  \end{block}
   
 \end{frame}

    
%-------------------------------------------------------------------------------------------------------------------------------------
\section{PDE models}
%-------------------------------------------------------------------------------------------------------------------------------------
%------------------------------------------------------------------------------------------------------------------------------
\begin{frame}

\begin{block}{examples}
\begin{itemize}
    \pause
      \item  $u_{tt}(x,t) - c^2 u_{xx}(x,t) =0$, the Wave equation
          \pause
             \item  $u_{xx}(x,y) + u_{yy}(x,y) =0$, Laplace's equation
                  \pause
             \item  $u_{t}(x,t) = D u_{xx}(x,y)$, the Diffusion equation     
       \end{itemize}
  \end{block}
  
 \end{frame}

%-------------------------------------------------------------------------------------------------------------------------------------
%------------------------------------------------------------------------------------------------------------------------------
%\begin{frame}{Boundary and Initial Conditions}
\begin{frame}

\begin{block}{Heat Flow}
\begin{itemize}
    \pause
      \item  A laterally insulated metal bar of length $L$ with zero temperature at both ends.  Heat flows in the axial direction with the temperature given by $u(x,t)$.  Suppose that initially the heat distribution is given by $u(x,0) = \phi(x)$.
          \pause
      \item The \tu{heat equation} 
      \begin{center}
            $u_t = k u_{xx}$.
      \end{center}
         \pause
      \item \tu{Boundary conditions} are given by
            \begin{center}
            $u(0,t)= 0$ and $u(L,t)= 0$
      \end{center}
               \pause
      \item \tu{Initial condition} is given by
            \begin{center}
            $u(x,0)= \phi(x)$ for $0 \leq x \leq L$.
      \end{center}
        \pause
      \item This model is referred to as an \tu{evolution equation} as it depends on time. 
      
       \end{itemize}
  \end{block}
  
 \end{frame}
%------------------------------------------------------------------------------------------------------------------------------
%\begin{frame}{Boundary and Initial Conditions}
\begin{frame}

\begin{block}{Heat Flow - modified}
\begin{itemize}
    \pause
      \item  A laterally insulated metal bar of length $L$ with \tu{insulated} edges at both ends.  Heat flows in the axial direction with the temperature given by $u(x,t)$.  Suppose that initially the heat distribution is given by $u(x,0) = \phi(x)$.
          \pause
      \item The \tu{heat equation} 
      \begin{center}
            $u_t = k u_{xx}$.
      \end{center}
         \pause
      \item \tu{Boundary conditions} are instead given by
            \begin{center}
            $\ds \frac{\d u}{\d x}\biggr |_{(0,t)}= 0$ and $\ds \frac{\d u}{\d x}\biggr |_{(L,t)}= 0$
      \end{center}
               \pause
      \item these are no flux boundary counditions
       \end{itemize}
  \end{block}
  
 \end{frame}
%-------------------------------------------------------------------------------------------------------------------------------------
%\begin{frame}{Boundary and Initial Conditions}
\begin{frame}

\begin{block}{Laplace's equation}
\begin{itemize}
    \pause
      \item  Consider the steady state temperature $u(x,y)$ on a bounded two dimensional domain $\Omega$.  Along the boundary, $\d \Omega$, there is an imposed temperature distribution given by $f(x,y)$.
          \pause 
      \begin{center}
            $u_{xx}+u_{yy}=0$ where $(x,y) \in \Omega$.
      \end{center}
         \pause
      \item subject to the \tu{boundary conditions} 
            \begin{center}
            $u(x,y)= f(x,y)$ on  $(x,y) \in \d \Omega$  
      \end{center}
      \pause
      \item A  constant boundary condition is called a \tu{Dirichlet condition}.
               \pause
      \item Such a model is time independent  and called a \tu{steady state} or an \tu{equilibrium model}.        
      \pause
       \item In addition, such spatial problems are referred to as a \tu{boundary value problems.}
      
       \end{itemize}
  \end{block}
  
 \end{frame}

%-------------------------------------------------------------------------------------------------------------------------------------
\begin{frame}

\begin{block}{General Form of 2nd-order evolution equation PDE}

\begin{itemize}
    \pause
      \item  A 2nd-order evolution PDE in one spatial and one time variable can be written as 
          \pause 
      \begin{center}
            $G(x,t,u,u_x, u_t,u_{xx}, u_{tt}, u_{xt})=0$ on $x \in I$ and $t>0$
      \end{center}
  
         \pause
      \item order of a PDE is the order of the highest derivative
            \pause
      \item a PDE is said to be \tu{linear} if the operator $G$ is linear function of $u$ and its derivatives.
               \pause
      \item If $G$ is not linear in $u$ and its derivatives, then PDE is \tu{nonlinear}. 
      
       \end{itemize}
  \end{block}
  
  
  
 \end{frame}

%-------------------------------------------------------------------------------------------------------------------------------------
\begin{frame}

\begin{block}{Linear Operators}
Let $L$ denote an operator that acts on $u \in D$.  Then $L$ is said to be linear provided
\begin{itemize}
    \pause
      \item  $L(u+v) = Lu + Lv$, where $u,v \in D$.
         \pause
      \item  given $c$ is a constant, $L(cu) = c Lu$
         
\end{itemize}
  \end{block}
  
           \pause
  
  \begin{block}{example}
               Show that $L = \frac{\d}{\d t} - k \frac{\d^2}{\d x^2}$ is a linear operator acting on $u(x,t)$ where $x \in I$ and $t>0$.
   \end{block}
   

   
 \end{frame}

%-------------------------------------------------------------------------------------------------------------------------------------
\begin{frame}

\begin{block}{Homogeneous vs nonhomogeneoous}

Consider a PDE given by 
      \begin{center}
            $G(x,t,u,u_x, u_t,u_{xx}, u_{tt}, u_{xt})=0$ on $x \in I$ and $t>0$
      \end{center}

\begin{itemize}
    \pause
      \item  a PDE is said to be \tu{homogeneous} provided every term contains a $u$ or a derivative of $u$.
               \pause
   \item  the PDE is said to be \tu{nonhomogeneous} if there is a term depending only upon the independent variable $x$ and $t$.    
\end{itemize}

\end{block}
  

 \end{frame}
 %-------------------------------------------------------------------------------------------------------------------------------------
\begin{frame}

\begin{block}{note}
       A PDE is linear if it is of the form:
      \begin{center}
                    $a_0(x,t) u + a_{1,0}(x,t) u_x  + a_{0,1}(x,t) u_t + a_{2,0}(x,t) u_{xx} +a_{0,2}(x,t) u_{tt}+ a_{1,1}(x,t) u_{xt}=g(x,t)$ 
      \end{center}
      
      \begin{itemize}
        \pause
          \item  if a PDE is linear and \tu{homogeneous} then $g(x,t)=0$.
         \pause
        \item    if a PDE is linear and \tu{nonhomogeneous} then $g(x,t) \neq 0$.
        \end{itemize}
        
 \end{block}         
     
           \pause

\begin{block}{classify the following}

  \begin{enumerate}
      \item  $u_t+u u_{xx}=0$
 
      \item $u_t +3 x u_{xx} = t x^2$
 
      \item $u_t - \sin(x^2t)u_{xt} = 0$        
      
       \item $u_{tt} -u_x + \sin(u) = 0$           
  \end{enumerate}
\end{block} 

   

 \end{frame}
 %-------------------------------------------------------------------------------------------------------------------------------------
 \begin{frame}{Importance of homog linear PDEs}

\begin{block}{Superposition Principle}
            Given that $\ds \{ u_i \}_{i=1}^n$ are solutions to the linear homog PDE given by $Lu=0$, \pause
            then the linear combination $\ds u(x,t) = \sum_{i=1}^n c_i u_i(x,t)$ also solves the PDE.
\end{block}
  
           \pause
  
\begin{block}{extension}
           Suppose that $u(x,t, \xi)$ solves the linear homog PDE $Lu=0$ for all $\xi$ on an interval $J$,  then  $\ds u(x,t) = \int_J c(\xi) u(x,t,\xi) d\xi$ solves the PDE.
\end{block}

  
   
 \end{frame}


%-------------------------------------------------------------------------------------------------------------------------------------
\begin{frame}

\begin{block}{example}
Verify that $u_1(x,t)= x^2 +2t$ is a solution to 
      \begin{center}
                    $u_t - u_{xx}= 0$ 
      \end{center}
      
  \end{block}
  
           \pause
  
\begin{block}{problem}
Verify that $u_2(x,t)= e^{-t} \sin(x)$ is also a solution to 
      \begin{center}
                    $u_t - u_{xx}= 0$ 
      \end{center}
      
  \end{block}

  
   
 \end{frame}
%-------------------------------------------------------------------------------------------------------------------------------------
\begin{frame}

\begin{block}{example i}
Solve $u_x= t \sin(x)$.
 \end{block}
  
           \pause
By direct integration wrt $x$, 
\begin{eqnarray*}
      u(x,t) & = & \int t \sin(x) \d x  \\    \pause
              & = & -t \cos(x) 
                        + \psi(t),   \pause 
                         \text{where } \psi(t) \text{ is an arbitrary functions of } t. 
\end{eqnarray*}                        
\pause

     \begin{block}{example ii}
Solve for $u(x,t)$ where $u_{tt} -4u = 0$.
 \end{block}
 
  \begin{itemize}
            \pause
            \item This is a 2nd-order ODE in $t$ with $x$ a free variable.              
            \pause
            \item The fundamental set of the ODE is $\{ e^{-2t}, e^{2t} \}$.  
            \pause
            \item The general solution to the PDE is then 
            \pause
           \begin{center}
                   $ u(x,t) = \phi(x) e^{-2t}+ \psi(x) e^{2t}$
            \end{center}
            where $\phi(x)$ and $\psi(x)$ are arbitrary functions of $x$. 
   \end{itemize}
 \end{frame}
%-------------------------------------------------------------------------------------------------------------------------------------
 \begin{frame}{Homework}
    \begin{itemize} 
             \item  Section 1.1: 1, 2, 3, 5, 7, 11, 12/13 (a, b, e)
    \end{itemize}
 
 \end{frame}
%--------------------------------------------------------------------------------------------------------------------

%--------------------------------------------------------------------------------------------------------------------
\section{Conservation Models}
%--------------------------------------------------------------------------------------------------------------------
 \begin{frame}{Conservation Law}
\begin{center}
\vspace{-3.2cm}
\includegraphics[width=0.5\textwidth]{fig_conservation.pdf}
\end{center}
 \vspace{-2.5cm}
 \begin{itemize}
     \pause
     \item Consider a tube with cross-sectional area $A$.  The lateral sides are insulated and physical quantities vary only in the $x$ direction and time $t$.  All quantities are constant over a cross section.  
     \pause
     \item Let the state variable $u(x,t)$ denote the density of a given quantity (mass, energy, animals, automobiles, $\ldots$)
          \pause
     \item The \tu{flux} $\phi(x,t)$ represents the amount of the quantity crossing the section at $x$ position at time $t$ per unit area.
     \pause
     \item By convention, flow to the right is positive, to the left is negative.  
     \pause
     \item Thus, $A \cdot \phi(x,t)$ represents the total total quantity crossing the section at $x$ position at time $t$.
     \pause
     \item  Let $f(x,t)$ denote the rate at which the quantity is created or destroyed.  
          \pause
     \item  If $f > 0$, called a source, if $f < 0$, called a sink.
 \end{itemize}
 
 \end{frame}
%--------------------------------------------------------------------------------------------------------------------
 \begin{frame}{}
\begin{center}
\vspace{-2.7cm}
\includegraphics[width=0.6\textwidth]{fig_conservation.pdf}
\end{center}
 \vspace{-2.7cm}
 \begin{block}{Conservation Law}
  \begin{itemize}
     \pause
     \item We wish to construct a mathematical formulation to track the change in the total quantity in the tube between $x=a$ and $x=b$.
       \pause
       
 \end{itemize}
\begin{center}
       $ \ds \frac{ d }{ dt}  \(  \int_a^b u(x,t) A dx  \) = \pause
            A \cdot \phi(a,t) - A \cdot \phi(b,t) \pause
             + \int_a^b f(x,t) A dx$
\end{center}
\pause
This is the fundamental conservation law.   
\end{block}
  
 \end{frame}
 %--------------------------------------------------------------------------------------------------------------------
 \begin{frame}{}
\begin{center}
\vspace{-2.4cm}
\includegraphics[width=0.5\textwidth]{fig_conservation.pdf}
\end{center}
 \vspace{-2.4cm}

 
 \pause 
 
  \begin{block}{Conservation Law}
  \begin{itemize}
     \pause
     \item Suppose $u$ and $\phi$ have continuous first derivatives then   \pause
       

 
 \vspace{-3ex}
\begin{eqnarray*}
      \ds  \frac{ d }{ dt}  \(  \int_a^b u(x,t) A dx  \)  & = &  \int_a^b u_t(x,t) A dx  \\ \pause
       \text{and } A \cdot \phi(b,t) - A \cdot \phi(a,t)   & = &  A \int_a^b \phi_x(x,t) dx \\ \pause
        \implies     \int_a^b u_t(x,t) A dx  & = &   \int_a^b \( -\phi_x(x,t) + f(x,t) \) A dx
\end{eqnarray*}
\pause
 \vspace{-1.5ex}
\item $\implies u_t(x,t) + \phi_x(x,t)= f(x,t),$ also called the \tu{fundamental conservation law.  }

 \end{itemize}
 
\end{block}
 
 
 \end{frame}
%--------------------------------------------------------------------------------------------------------------------
 \begin{frame}{Method of Characteristics}

 \pause 
 
  \begin{block}{Advection}
  \begin{itemize}
     \pause
     \item If the flux is proportional to the density, $\phi(x,t) = c u(x,t) $, is called an advection model.  
     \pause
     \item Paired with the fundamental conservation law, we obtain the  \tu{advection equation} 
     \begin{center}
            $ u_t + c u_x = f(x,t)$.
     \end{center}
     \pause
     \item Consider no sources or sinks, that is, $f(x,t) =0$, so that 
          \begin{center}
            $ u_t + c u_x =0$.
     \end{center}
          \pause
     \item Observe $u(x,t) = F(x-ct)$ is a solution for any differentiable $F$.
           \pause
     \item This is a right-traveling wave.      
                \pause
     \item other descriptives for bulk motion is \tu{transport} or \tu{convection}.
\end{itemize} 
 
 
\end{block}
 
 
 \end{frame}
%--------------------------------------------------------------------------------------------------------------------
 \begin{frame}{}

 \pause 
 
  \begin{block}{example}
            \begin{center}
            $ u_t + c u_x =0$ \text{ with initial condition: } $u(x,0) = u_0(x)$
     \end{center}
 
  \begin{itemize}
     \pause
     \item From previous example, it follows that $u(x,t) = u_0(x -ct)$ is the solution.  
     \pause
     \item Initial signal moves to the right with speed $c$.  
     \pause
     \item Alternatively consider the signal moving along the family of parallel straight lines along $\xi = x - ct$.  
     \pause
      \item     These lines are called the \tu{characteristics}.     
     \pause
      \item    $u$ is constant along the characteristics.         
        \end{itemize} 
 
 
\end{block}
 
 
 \end{frame}
%--------------------------------------------------------------------------------------------------------------------
 \begin{frame}{}

 \pause 
 
  \begin{block}{General Advection Equation with Characteristics}
  
            \begin{center}
            $ u_t + c u_x +au = f(x,t)$ 
           \end{center}
 \vspace{-3ex}
 
  \begin{itemize}

     \pause
     \item  The above model has advection, decay, and a reaction term.
          \pause
     \item Introduce the \tu{characteristic coordinates}
            \begin{center}
            $ \xi = x-ct$ and $\tau = t$ 
     \end{center}
     \pause
     \item  Want to solve in terms of $U(\xi, \tau)$.       
     \pause
     \item Using the chainrule 


\vspace{-1ex}
     \begin{minipage}{.4\textwidth} %
     \begin{eqnarray*} 
            u_t & =  & \frac{ \d U}{\d \xi}   \frac{ \d \xi}{\d t}  +  \frac{ \d U}{\d \tau}   \frac{ \d \tau}{\d t}   \\ \pause
                  & =  &U_{\xi}  (-c)  + U_{\tau}  (1)    \\ \pause
                   & =  &-c U_{\xi}   + U_{\tau}  \pause
      \end{eqnarray*}
      \end{minipage} %
  \begin{minipage}{.4\textwidth} %
     \begin{eqnarray*} 
            u_x & =  & \frac{ \d U}{\d \xi}   \frac{ \d \xi}{\d x}  +  \frac{ \d U}{\d \tau}   \frac{ \d \tau}{\d x}   \\ \pause
                  & =  &U_{\xi} (1)  + U_{\tau}  (0)    \\ \pause
                   & =  &U_{\xi}  
      \end{eqnarray*}
 \end{minipage} %
 
     \pause
     \item Substituting back into the PDE to obtain, 
    \begin{eqnarray*} 
            u_t + c u_x +au & =  &  f(x,t)   \\ \pause
         ( -c U_{\xi}   + U_{\tau} ) + c U_{\xi}   + a U     & =  & f(\xi+c\tau, \tau)    \\ \pause
               U_{\tau} + a U  & =  & F(\xi,\tau)   \pause
      \end{eqnarray*}
   \item an ODE in $\tau$. 

        \end{itemize} 
 
 
\end{block}
 
 
 \end{frame}
%--------------------------------------------------------------------------------------------------------------------
\begin{frame}

\begin{block}{work along example}
  Find the general solution to 
  \begin{center}
          $u_t+2u_x-u=t.$
  \end{center}
\end{block}

\end{frame}
%--------------------------------------------------------------------------------------------------------------------
%--------------------------------------------------------------------------------------------------------------------
\begin{frame}

\begin{block}{general reaction-advection PDE}

\begin{center}
         $u_t + c(x,t) u_x  =  f(x,t,u) $
  \end{center}
  
\end{block}

\pause

\begin{block}{non-constant advection}
  Solve $dx/dt = c(t)$.  Let $\xi(x,t) =C$ where $\xi$ is the general solution to the ODE $dx/dt = c(xt)$.  \pause

\begin{itemize}
\item example i: Determine the general solution to
   \begin{center}
          $u_t+2tu_x=0$
  \end{center}

\item example ii:
 Determine the solution to 
  \begin{center}
          $u_t+2u_x=0.$
  \end{center}
  subject to the initial and boundary conditions.  
  \begin{center}
          $u(x,0)=e^{-x}$ and $u(0,t)= (1+t^2)^{-1}$
  \end{center}  
  
\end{itemize}

\end{block}


\end{frame}
%--------------------------------------------------------------------------------------------------------------------
\begin{frame}

\begin{block}{nonlinear advection}

       \begin{itemize}
       \pause
       \item Consider the flux to be dependent on $u$,  that is, $\phi(u)$.
       \pause
       \item By the chain rule,
         \begin{center}
          $\ds \frac{ d \phi(u) }{dx} = \phi'(u) u_x $
        \end{center}
       \pause
       \item Denoting $\phi'(u) = c(u)$, then the associated PDE is
           \begin{eqnarray*}
           u_t + c(u) u_x & = & 0 \\
                             u(x,0) & = & \phi(x)
        \end{eqnarray*}
       \pause
       \item wave speed depends on the density, $u$        
       \end{itemize}
\end{block}
\end{frame}
%--------------------------------------------------------------------------------------------------------------------
\begin{frame}
\pause
\begin{block}{solution technique}

       \begin{itemize}
       \pause
       \item The \tu{characteristic curves} are given by 
                \begin{center}
          $\ds \frac{ dx }{dt} = c(u)$
        \end{center}
       \pause
       \item along the characteristic curves $x=x(t)$, $u$ is a constant.  Thus it follows that,
         \begin{eqnarray*}
             \ds    \frac{ d^2x }{dt^2}   & = &    \frac{ d c(u )}{dt}          \pause
                        =  c'(u) \frac{ du}{dt}   =  \pause     
                        c'(u) \[ u_x  \frac{ dx }{dt}  + u_t  \frac{ dt }{dt}  \]   \\   \pause
                     &=  &c'(u) \[ u_x  c(u)  + u_t \] = c'(u) \[ 0 \] =0 
        \end{eqnarray*}
       \pause
       \item Thus the characteristic curves occur along straight lines!
                \begin{center}
          $\ds \frac{ dx }{dt} = c(u(\xi,0)) $ \pause
                                      $ =c(\phi(\xi))$ \pause
                                     $ \implies x(t) = c(\phi(\xi)) t + \xi $
               \end{center}
               \pause
       \item The solution to the IVP is $u(x,t) = \phi(\xi)$ where $\xi$ is defined implicitly above.
       
       \end{itemize}
       
\end{block}


\end{frame}
%--------------------------------------------------------------------------------------------------------------------
\begin{frame}
\pause
\begin{block}{example}
Solve the IVP,
\begin{center}
$u_t + u u_x =0 \quad u(x,0)=\phi(x) = 2 - x [ H(x) -H(x-1) ] - H(x-1)$.
\end{center}
\end{block} 
\pause
\begin{block}{solution technique}

       \begin{itemize}
       \pause
       \item the front travels more slowly than the back of the wave
              \pause
    %   \item for $t<1$, the solution  $u(x,t) = \phi( \xi)$ where
    %           \begin{center}
       %                              $x(t) = c(\phi(\xi)) t + \xi $
     %          \end{center}      
               
        \end{itemize}
       
\end{block}


\end{frame}
%--------------------------------------------------------------------------------------------------------------------
\begin{frame}
\pause
\begin{block}{Traffic Flow}

Consider traffic moving in a single spatial direction with a car density given by $\rho(x,t)$ (units in cars per km).  This can be modeled by the PDE  
\begin{center}
$\rho_t + (\rho v)_x =0 $.
\end{center}
where $v(x,t)$ is the local car speed.  Thus $\phi = (\rho v)$ is the flux. \pause

\begin{itemize}
      \item Reasonable to consider car speed to depend on local traffic.  \pause
      \item A model could be
      \begin{center}
                 $\rho v =  \rho v_M (1 - \frac{\rho}{\rho_J})  $.
        \end{center}
        \pause
        \item $v_M$ is maximum velocity.  \pause
         \pause
        \item $\rho_J$ is the density where a traffic jam occurs.  
                 \pause
        \item Change of variables for a scaled density:  $ u(x,t) =\ds  \frac{\rho(x,t)}{\rho_J}$.  
         \pause
              \item The model then becomes
      \begin{center}
                 $u_t+ (1-2u) u_x = 0 $.
        \end{center}
\end{itemize}

\end{block}






\end{frame}
%--------------------------------------------------------------------------------------------------------------------

%--------------------------------------------------------------------------------------------------------------------
\section{Diffusion}
%--------------------------------------------------------------------------------------------------------------------
\begin{frame}{Diffusion}
\pause
\begin{block}{}

\begin{itemize}
   \pause
      \item Consider $u(x,t)$ to represent the density of a gas.  
      \pause
      \item The particles exhibit random motion with ccllsions.
       \pause
\item Fundamental Conservation Law:   $ u_t(x,t) + \phi_x(x,t)= f(x,t)$ 
\end{itemize}

\end{block}


\pause
\begin{block}{Fick's Law}

\begin{itemize}
      \pause
      \item Movement is from higher to lower concentrations. 
             \pause
      \item The steeper the concentration gradient, the greater the flux.
      \pause
\item Fick's Law:  $\quad  \phi_(x,t)= -D u_x(x,t),$
\end{itemize}

\end{block}




\end{frame}
%--------------------------------------------------------------------------------------------------------------------
\begin{frame}{Diffusion Equation}
\pause
\begin{block}{}

\begin{center}
     $ u_t(x,t) = Du_{xx}(x,t)$ 
\end{center}

\end{block}


\pause

\begin{center}
\vspace{-3.2cm}
\includegraphics[width=0.5\textwidth]{fig_conservation.pdf}
\end{center}
 \vspace{-2.5cm}
\pause




\end{frame}
%--------------------------------------------------------------------------------------------------------------------
%--------------------------------------------------------------------------------------------------------------------
\section{Diffusion and Randomness}
%--------------------------------------------------------------------------------------------------------------------
%--------------------------------------------------------------------------------------------------------------------
\section{Vibrations and Acoustics}
%--------------------------------------------------------------------------------------------------------------------


\end{document}


